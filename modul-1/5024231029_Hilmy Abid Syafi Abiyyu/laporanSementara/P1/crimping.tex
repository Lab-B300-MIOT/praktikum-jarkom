\section{Pendahuluan}
\subsection{Latar Belakang}
\paragraph{} 
Jaringan komputer adalah sistem yang menghubungkan dua atau lebih perangkat komputasi seperti komputer, laptop, smartphone, dan perangkat lainnya untuk berbagi data, informasi, atau sumber daya. Jaringan komputer memiliki peran penting dalam menghubungkan berbagai perangkat untuk bertukar data. Untuk membuat sebuah jaringan komputer, dibutuhkan pemahaman mengenai konfigurasi jaringan IPv4, termasuk teknik crimping kabel, perancangan topologi jaringan, serta pengaturan routing statis dan dinamis. Teknik crimping penting dipelajari untuk memastikan koneksi fisik antar perangkat berjalan dengan baik. Topologi jaringan membantu menentukan struktur dan efisiensi komunikasi data.

Routing statis dan dinamis adalah inti dari pengaturan jalur data dalam jaringan. Routing statis memberikan kontrol penuh dengan sisi buruknya yaitu kurang fleksibel. Routing dinamis memungkinkan penyesuaian otomatis terhadap perubahan jaringan. Tujuan dilaksanakannya praktikum ini agar dapat memahami dan mengaplikasikan crimping serta routing jaringan IPv4.
\subsection{Dasar Teori}
Bagian ini memuat teori-teori dasar yang mendukung pelaksanaan praktikum. Penjelasan mencakup konsep teknis, nama istilah, serta prinsip ilmiah yang relevan. Tujuannya adalah untuk memberikan pemahaman mendalam sebelum praktikum dilakukan.

%===========================================================%
\section{Tugas Pendahuluan}
Bagian ini memuat teori-teori dasar yang mendukung pelaksanaan praktikum. Penjelasan mencakup konsep teknis, nama istilah, serta prinsip ilmiah yang relevan. Tujuannya adalah untuk memberikan pemahaman mendalam sebelum praktikum dilakukan.
\begin{enumerate}
	\item jawaban
	\item jawaban
	\item jawaban
\end{enumerate}